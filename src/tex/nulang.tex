\documentclass[oneside]{book}

% colours
\usepackage[dvipsnames]{xcolor}
\definecolor{bg}{rgb}{0.95,0.95,0.95}
\definecolor{exbg}{rgb}{0.80,0.95,0.95}
\definecolor{emphcolor}{rgb}{0.5,0.0,0.0}
\newcommand{\empha}{\bf\color{emphcolor}}
\usepackage[framemethod=TikZ]{mdframed}

% theorem like environments
\usepackage{amsthm}
\theoremstyle{plain}
\newtheorem{example}{Example}
\theoremstyle{definition}
\newtheorem{definition}{Definition}
\theoremstyle{plain}
\newtheorem{lemma}{Lemma}
\newtheorem{theorem}{Theorem}
\newtheorem{notation}{Notation}

\mdfdefinestyle{MyFrame}{innerleftmargin=20pt}
\newenvironment{myexample}%
  {\begin{mdframed}[style=MyFrame,backgroundcolor=exbg]}%
  {\end{mdframed}}
\usepackage{parskip}
\usepackage{minted}
\usepackage{caption}
\usepackage{amsmath}
\usepackage{amssymb}
\usepackage{amscd}
\usepackage{graphicx}
\usepackage{framed}
\usepackage{color}
\usepackage{tcolorbox}

\definecolor{gray}{gray}{0.3}
\definecolor{darkgreen}{rgb}{0,0.55,0}
\definecolor{purple}{rgb}{0.5,0,1}

\newtcolorbox{xcomment}{colback=red!5!white,colframe=black!60!white}
\newtcolorbox{ycomment}{colback=green!5!white,colframe=black!60!white}

\def\Set{\mathit{Set}}
\def\Nat{\mathit{Nat}}
\def\Hom{\mathit{Hom}}
\def\ALU{\mathit{ALU}}
\def\MMU{\mathit{MMU}}

% minted
\usemintedstyle{friendly}
\setminted{bgcolor=bg,xleftmargin=20pt}
\usepackage{hyperref}
\hypersetup{pdftex,colorlinks=true,allcolors=blue}
\usepackage{hypcap}

\usepackage[backend=biber]{biblatex}
\addbibresource{../src/tex/nubib.bib}

\title{Maclean\\ A new language for the modern world}
\author{John Skaller}
\begin{document}
\maketitle
\tableofcontents
\chapter{Rationale}
For many years binary byte addressable von Neumann machines dominated the computing world.
However the world has changed, with increasing demands for security and privacy and
trusted access to computing power required in an untrustworthy environment.

\begin{xcomment}
With the rise of cryptography and the invention of the block chain, new capabilities
have arisen, along with new requirements. The established von Neumann architectures
are no longer suitable.
\end{xcomment}

Block chain operations are basically microservices operating in an untrusted environment,
where proof that a server faithfully executed a program is required, in addition to the
older requirement that the program be correct. Program correctness has been addressed
by advanced type systems, but the new breed of system, often represented by a circuit
or a virtual machine, no longer uses the same basic types as the traditional signed
and unsigned integers of 8, 16, 32 or 64 bits.

Instead, finite fields are required for cryptographic reasons, addressing models vary,
and performance is now dominated by the cost of proof generation, rather than the
cost of execution.

We still need strong static typing for proofs of correctness, but computation and
optimisation must now deal with both execution, and proof generation. Performance
of virtual machines is considerably degraded so that the proof generation becomes
tractable.

\begin{xcomment}
Behind the birth of the crypto world, other new requirements arose. 
Clearly with the internet itself, there is a requirement for distributed,
concurrent processing, along with a need for sophisticated communications.
Yet whilst type systems for single threaded local memory access are
reasonably advanced there is only flegdling support for typing
communications, and only rudimentary understanding of concurrent
processes (even at the local level).
\end{xcomment}

Today, early vectorisation in supercomputers, primarily to compute
inner prodcts rapidly, has be superceded by more general and 
capable GPUs, capable of running thousands of threads concurrently.
Although these devices were originally for improving graphics display
rendering speeds, they have rapidly been adopted for more general
applications amendable to parellel processing. Yet mostly they're
programmed with a separate language than the CPU, and with very
little type safety.

\begin{ycomment}
In addition, the ad hoc nature of older languages is no longer tolerable,
instead we need a system soldidly based in algebra: ring theory and finite field
theory providing the basic types, and category theory providing ways to
construct new types from the base, abstraction, polymorphism, and modularity.
\end{ycomment}


\chapter{Category Theory}
\begin{definition}
A {\em category} is a directed multi-graph whose vertices are called {\em objects,}
and whose edges are termed {\em arrows} or {\em morphisms},
together with an equality operator for both objects and arrows,
and a partial binary operator called {\em composition} with three 
constraints:
\begin{enumerate}
\item {\em Closure:} If $A,B,C$ are objects, and $f:A\rightarrow B$ is an arrow
from A to B, and $g:B\rightarrow C$ is an arrow from B to C,
then there exists an arrow $h:A\rightarrow C$ such that
$$h = g \circ f = f \odot g$$
where $\circ$ is composition operator in forward notation and $\odot$
is the (vastly prefered) composition operator in reverse notation.
\item {\em Identity:} For every object $X$ there is a unique arrow $1_X:X\rightarrow X$
called an {\em identity} arrow, such that for all arrows $f:A\rightarrow X$
$$f \odot 1_X = f$$
and for all arrows $g:X\rightarrow C$
$$1_X \odot g = g$$
\item {\em Associativity:} If $f\rightarrow B$, $g:B\rightarrow C$ and $h:C\rightarrow D$ then
$$(f\odot g) \odot h = f\odot (g \odot h)$$
which as usual means the parentheses may be elided without ambiguity.
\end{enumerate}
The head object of an arrow is called the {\em domain} and the tail
object the {\em codomain}.

Note that due to the bijection between objects and identity arrows
we can throw out the objects altogether; such a category is called
an {\em arrows only} category. 
\end{definition}

\begin{example}
The collection of all small sets and functions between them is 
the category {\em Set}.  In this case the objects are sets,
and sometimes called types; the arrow are functions.
\end{example}

\begin{example}
For any directed graph, the collection of
all paths on the graph is a category; provided we add, if necessary, an identity
arrow for each vertex.
\end{example}

\begin{definition}
A {\em subcategory} of a category is a subset of objects and arrows
which forms a category. This means, objects and identity arrows
correspond, and the set of arrows must be closed under composition.
\end{definition}

\begin{definition}
If $D$ and $C$ are objects of a category, the set of all arrows 
from $D$ to $C$ is called the {\em hom set} from $D$ to $C$
and is written
$$\Hom(D,C).$$
\end{definition}

\begin{definition}
The {\em dual}, or {\em opposite} of a category $X$ is the same
category with all the arrows reversed, usually written $X^{\mathit{op}}$.
\end{definition}

\begin{definition}
A {\em covariant functor} is a structure preserving map
$F:A\rightarrow B$ from
categopry $A$ to category $B$, which maps objects to objects and
arrows to arrows and which preserves composition; that is,
$$F(f\odot_A g) = F(f) \odot_B F(g)$$
where $\odot_A$ is the composition operator on $A$, 
and $\odot_B$ is the composition operator on B.
\end{definition}
Visually this means triangles map to (possibly degenerate) triangles.

\begin{definition}
A {\em contravariant functor} reverses the arrows, so that
if $f:A\rightarrow B$ then $F(f):F(B)\rightarrow A$ and so
the structure preservation rule becomes 
$$F(f\odot g) = F(g) \odot F(f)$$
\end{definition}
A contravariant functor is a covariant functor from the opposite
category.

\begin{definition}
If $A$ and $B$ are categories, the {\em product} category, written $A\times B$
consists of pairs of objects and arrows from $A$ and $B$. The definition
can be extended to any finite number of categories.
\end{definition}

A functor from a pair of categories is sometimes called a {\em bifunctor}.
In computer science, a functor is also known as a {\em type constructor}
or simply a {\em polymorphic type}.
\chapter{Type System}
\section{Theory}
Let $\Set$ be the usual distributive category consisting of all small sets and functions.
Let $\Set_n$ be the full subcategory containing only sets of size less than $n$.

Let $\Nat$ be the full subcategory of $\Set$ consisting
of sets of integers $\mathit{N}_n=[0,n)$ for some $n$, and let
$\Nat_n$ be the full category consisting of the 
sets $N_k$ for $k\in \mathit{N}_n$.

\begin{figure}[h]
\includegraphics[width=0.8\textwidth]{../../IMG_0224.jpg}
\caption{Categorical inclusion diagram}
\label{fig:categorical inclusion}
\end{figure}

The objects of these categories are sets which we will also call {\em types.}

The category with no objects is called {\em Void}, with one object, the empty set,
is called {\em Unit}. The category $\mathit{Nat_2}$ has two objects, the empty set
and a set containing only 0, and is also called {\em Bool}.

The empty set $\mathit{N_0}$ is also called {\em void}, and the set $\mathit{N_1}$
with just $\{0\}$ {\em unit}, the set with two values, 0 and 1 is called {\em bool}.

The number of objects in a finite category is given by a the function {\em size}
which is overloaded to also given the number of values of a set, and is written
\begin{equation}
\mathrm{size}(X) = |X|
\end{equation}
so that in particular
\begin{equation}
|N_n|=n
\end{equation}

Note: it is important to understand that the type $N_3$ in $\Nat_4$ is the {\em same}
type as found in $\Nat_5$. It's not a different type because $\Nat_4$ is a subcategory
of $\Nat_4$. Similarly, an integer $3$ in $N_4$ is the {\em same} integer as the $3$
in $N_5$ even though the type appears different this is not so, because again, they're
subtypes. So what this means is that for example $3$ has type $N_4$ and type $N_5$ and
type $N_6$ etc. In other words, the embedding is the identity map.

It's usual in algebra to use embeddings: for example, the integers are embedded
in the rationals, but an integer $3$ is {\em not the same} as the rational $3/1$ 
it corresponds to.
 
 
The fundamental theorem of our construction is
\begin{lemma}
Every object $X$ of size $k<n$ in $\mathit{Set_n}$ is isomorphic to the unique object $N_k$
in $\mathit{Nat_n}$. 
\end{lemma}

The basic idea is that we have a functor 
\begin{equation}
\mathrm{Rep_1}: \Nat_n\rightarrow \Set_n,
\end{equation}
where $\mathrm{Rep}_1$ is an isomorphism.

To perform a computation in the image of the functor in $\Set_n$ 
we first apply the inverse functor, perform the calculation using integers in $\Nat_n$
and then apply the functor to go back from the representation space to the represented space.

However this is not enough because for example we can have two distinct objects in $\Set_n$
of the same size which "behave differently" so we would need a second functor 
\begin{equation}
\mathrm{Rep_2}: \Nat_n\rightarrow \Set_n
\end{equation}
which is also an isomorphism. In fact, the same object can be used in different ways,
and so "behave differently" in different contexts.

However, we cannot have this second functor model operations on a single objects
behaving "the same way" in each context give different results, these functors
must be constrained so that it does not matter which representation is used, 
we get the same answer. This leads to the following core theorem:
\begin{theorem}
Any two representations must be naturally equivalent, in other words,
the representation functors must be related by a natural transformation.
\end{theorem}



\subsection{Compact Tuple}
\begin{definition} {\em Compact Tuple}.
The compact tuple functor has kind 
$$\mathrm{Tup}: (Nat-N_0)^{\mathit{N_n}}\rightarrow \Nat,$$
where the index set is considered as a discrete set.
\footnote{the index structure is crucial and will be generalised later} 
It maps types
$$(N_{i_0}, N_{i_1}, ... , N_{i_{n-1}}) \mapsto N_m$$
where 
$$m=\prod_{j=0}^{n-1}i_j.$$ 

The value map is given by
$$(v_0, v_1, ... ,v_{n-1}) \mapsto
v_0  r_0 + v_1  r_1 + ... + r_{n-1}  v_{n-1}$$
where
$$r_k = \prod_{j=k+1}^{n-1}i_j$$
where the empty product is 1.
 
The projection functions of type 
$$\mathrm{prj}_k: N_m\rightarrow N_k$$
of the $k'th$ index is given by 
$$v \mapsto v / r_k \mod i_k$$
using Euclidean (integer) division.

The common notation for tuple types is
$$N_{i_0} \times N_{i_1} \times ... \times N_{i_{n-1}}$$
and values are usually written
$$v_0, v_1, ... ,v_{n-1}$$.

When it is clear, we will just write $n$ instead of $N_n$, for
example 5 instead of $N_5$.

Note that the result of the functor is {\em NOT} a type: this cannot be
emphasised enough. The result is a family of projections with the
same index as the functor.
\end{definition}

Note 
\begin{itemize}
\item Category theory explained \cite{products}
\item the domain categories have the empty set excluded because its setwise cartesian product
with a non-empty set is the empty set, and is not a categorical product due to the non-existence of 
a projection for the non-empty component.

\item
for a domain component of unit type $N_1$ a projection will be the unique
map $N_m\rightarrow N_1$ and the result will be
the value 0, independent of the constructed value $v$, since $x \mod 1$
always has value 0. In particular no state is required for storage of the
component and our encoding does not allocate any. Nevertheless the encoding
of the projection yields the correct result, namely 0.

\item
if the index set itself is empty, the product exists and is the unit type $N_1=\{0\}$
and there are no projections so the categorical requirement is satisfied in vaccuo.

\item
the indexing scheme is big-endian so a component with index 0 has the highest value.
This is un-natural mathematically but conforms to the usual Western representation
of numbers, lexicographical ordering, and the usual convention in programming languages.
Of course we use Arabic numbers, and Arabic is read right to left.

\item
product formation is not associative: associates are isomorphic but not equal;
they have different projections; however, they all have the same domain. For example
$$5 \times 3 \times 2 \neq (5 \times 3) \times 2 \neq 5 \times (3 \times 2)$$
have projections $30\rightarrow 5,\,30\rightarrow 3,\,30\rightarrow 2$ and 
$30\rightarrow 5\times 3,\,30\rightarrow 2$ and $30\rightarrow 5,\,30\rightarrow 3\times 2$,
respectively. 
\end{itemize}

The construction given is well known and is commonly called a {\em variadic number system}.


\begin{center}
\begin{tabular}{|l|r|r|r|r|}
\hline
\multicolumn{5}{|c|}{British Monetary System}\\
\hline
name & pounds & shillings & pence & farthings \\
$k$&0&1&2&3\\ 
$i_k$&  100 &  20 & 12 & 4 \\ 
$r_k$         &   960 & 48  & 4  & 1 \\ \hline

\multicolumn{5}{|c|}{Constructor}\\ \hline
$v_k$& 2     &  3  &  4  & 1 \\
$v_k\times r_k$ & 1920 & 144 & 16 &  1 \\ \hline
$v=\sum_0^3v_k\times r_k$&\multicolumn{3}{|c|}{}&2081\\ \hline
\multicolumn{5}{|c|}{Projections}\\ \hline
$v/r_k$& 2 & 43 & 520 & 2081 \\
$v/r_k\mod i_k$&2&3&4&1\\
\hline
\end{tabular}
\end{center}

For example in the British monetary system, 20 shillings is a pound, 
12 pence is a shilling, and 4 farthings is a penny. So v0=2 pounds, v1=3 shillings, v2=4 pence,
and v3=1 farthing is equal to 2 * (20 * 12 * 4) + 3 * (12 * 4) + 4 * (4) + 1 * (1) farthings.
We store 2081 farthings. We extract pounds as 2081 / (20 * 12 * 4) = 2, shillings as
(2081 / (12 * 4)) mod 20 = 3. pence as 2081 / (4) mod 12 = 5 and 2081 / (1) mod 4 = 1 farthing
where the formulae in parentheses are precisly the $r_k$ above. 

Note the first modulo is not required by construction, and the final division will 
always be by 1. The store of farthings is a compact product, and the extractors
of each denomination are projections. The general formulae and category theoretic
rationalisation reduces to primary school arithmetic!



\subsection{Compact Enumeration}
In order to understand compact enumerations with arguments we will first start 
with a special case, the unit sum. A type we would write as 1+1+1 can be thought
of as a 3 slot check box, where you mark 1 of the three slots with your choice.
The encoding of your choice will simply be its position, 2, 1 or 0.
In otherwords the type $N_3$ can be considered a unit sum.

Now suppose you have a check box that gives two choices, so that if you select
the first, you get 3 more choices, and if you select the second, you get 4 choices.
Thus can be written (1+1+1)+(1+1+1+1) or more simply 3+4, and is equivalent
to having 7 choices. In particular we can just use the value 6 for the first choice,
5 for the second choice, and so on down to 0. More precisely, values in the range
4..6 inclusive represent the first case and 3..0 the second case.

Therefore we can figure out which of the top level two cases are encoded
by testing if the stored value is in the inclusive range 4..6 (first case),
or 3..0 (second case).

If we find the first case we have to subtract 4 to get the summand.
otherwise we have the second case and the value is the answer.

It takes only a second to realise that the argument of a sum type
in our system is always a type with $k$ states; that is, $N_k$
for some $k$, even if it was constructed as a product, the formula
will work to find the argument.

Category theorists call this type constructor a sum or coproduct,
and programmers usually call them variants, however we will
follow Rust and just call them enumerations.

\begin{definition} {\em Compact Enumeration}.
The compact enumeration functor has kind 
$$\mathrm{Enum}: (Nat-N_0)^{\mathit{N_n}}\rightarrow \Nat,$$
where the index set is considered as a discrete set.
It maps types
$$(\mathit{N}_{i_0}, \mathit{N}_{i_1}, ... , \mathit{N_{i_{n-1}}}) \rightarrow \mathit{N}_m$$
where 
$$m=\sum_{k=0}^{n-1}i_k$$. 

The injection function index $k$ type $$\mathrm{inj}_k : N_{i_k} \rightarrow \Nat_m$$
constructs the enumeration as follows: 
$$v\mapsto v + s_k$$
where
$$s_k = \sum_{j=k+1}^{n-1}i_j$$
where the empty sum is 0.

Decoding an enumeration requires extracting both the index and injected
value. The index is given by the largest $k$ such 
that $v \geq s_k$ and the argument is then $v - s_k$. 
\end{definition}

Note
\begin{itemize}
\item
the ordering is big-endian, with the operand of the choice 
in position 0 being given the highest value.

\item
the required $k$ always exists because $s_{n-1}=0$.

\item
as for products we exlude $N_0$ from the input categories,
however for a different reason. There is a unique map from $N_0=\{\}$ 
to every set which is a perfectly good injection, but it cannot
be applied and so does not construct an enumeration, so has
no representation in our encoding scheme. A sum containing voids
is well defined but isomorphic to a sum with the voids removed
(except if it is the only one), so we will treat that isomorphismn
as an equality and exclude void cases.
\end{itemize}

\subsection{Compact Arrays}
If all the components of a compact product are the same type,
we can use a special array functor to construct it.
\begin{definition} {\em Compact Array}
The compact array functor has kind 
$$\mathrm{Arr}: (Nat-N_0)\times {\Nat}\rightarrow \Nat,$$

It maps types
$$(\mathit{N}_b, \mathit{N}_n) \mapsto \mathit{N}_m$$
where 
$$m=b \times n.$$

The value map is given by
$$(v_0, v_1, ... ,v_{n-1}) \mapsto
\sum_{k=0}^{n-1} v_k r_k$$
where
$$r_k = b^{n-1-k}$$
 
The projection functions of type 
$$\mathrm{aprj}: N_m \times N_n \rightarrow N_b$$
of the $k'th$ index is given by 
$$(v,k) \mapsto v / b^{n-1-k} \mod b$$
using Euclidean (integer) division.
\end{definition}

The type $N_b$ is the base type of the array, and the type $N_n$ is
the index type, the array will be length $n$. Not that unlike the
ordinary projection $\mathrm{prj}$, the $\mathrm{aprj}$ operator takes
an argument $k$ of the index type $N_n$, in other words the index
can be computed at run time. The reason is that the type of the
projection is independent of the index.

\subsection{Generalised Compact Arrays}
It seems a pity we cannot calculate the indices of a tuple at run time
because the tuple components all have different types.
Luckily there's a way to fix this! We simply unify projection codomains
into a coproduct type, namely, the categorical dual of the product type!
\begin{definition} {\em Generalised Compact Array}
The generalised compact array functor has kind 
$$\mathrm{GArr}: (Nat-N_0)\times {\Nat}\rightarrow \Nat,$$

It maps types
$$(\mathit{N}_{i_0}, \mathit{N}_{i_1}, ... , \mathit{N_{i_{n-1}}}) \mapsto \mathit{N}_p$$
where 
$$p=\sum_{k=0}^{n-1}{i_k}.$$ 

The value map is given by
$$(v_0, v_1, ... ,v_{n-1}) \mapsto
v_0  r_0 + v_1  r_1 + ... + r_{n-1}  v_{n-1}$$
where
$$r_k = \sum_{j=k+1}^{n-1}i_j$$
where the empty sum is 0.
 
The projection functions of type 
$$\mathrm{gprj}: N_m \times N_n \rightarrow N_p$$
of the $k'th$ index is given by 
$$(v,k) \mapsto (v / r_k \mod i_k) + s_k$$
where
$$s_k = \sum_{j=k+1}^{n-1}i_j$$
\end{definition}

Note that for example $A \times B \rightarrow A + B$, in other words the
projections all retun the sum type which is dual to the product type.
In particular the usual projection first returns the $k'th$ component 
which is then injected into the dual sum at the same position.
To retrieve the value could do the usual case analysis. however
since we know the index it can be retrieved immediately by
just subtracting $s_k$.


\subsection{Compact Coarrays}
A coarray, or repeated sum, is the dual of an array, that is,
it is a special case of the enumeration in which all the
enumerants have the same type. In this case the injection
functions can take a run time argument.  Now the sum of the
RHS components can be computed by a single multiply, and
instead of a linear search through subranges, we can 
use a binary chop to reduce the decoding time.

In fact, a different encoding could be used consiting of
a pair, the index and value which would admid constant
time encoding (by just using projections).

[add specs]

\section{The ALU}
We now have almost enough machinery to specify our first device.
\begin{definition}
An $\mathrm {ALU}_n$ is a device with two registers, an instruction register $F$
of size $k$
and a data register $A$ of size $n$. The instruction register contains an ALU implementation
specific encoding of a function $f:N_n\rightarrow N_n$ which is applied to the 
value in the $A$ register, which is replaced by the result of applying the function to it.
\end{definition}

A simple case of an ALU is a {\em synchronous} ALU which has a clock input.
It reads the instruction and data on the rising edge of the clock pulse,
and must write the result to the data register prior to the falling edge
of the clock.

An {\em asynchronous} ALU reads its input on an input clock pulse, writes
the result to the data register, and then emits an output clock pulse to
signify completion.

The key observation to be made, in either case, is that composition of functions is represented
temporaly by successive ALU operations.

Let $S$ be a subset of the functions of the type $N_n$.
The collection is said to be a {\em spanning set} if its transitive 
closure under composition is the set $\Hom(N_n,N_n)$, the set of all functions
from $N_n$ to $N_n$.

A spanning set is said to be a {\em basis} if the removal
of any one function would leave a non-spanning set.

An ALU is said to be {\em absolutely complete} if the set of instructions
it accepts map to the set of all functions of the type.
Only very small ALUs can be absolutely complete, since the number of
functions from a set of size $n$ to itself is $n^n$.

An ALU is seid to be {\em temporaly complete} or just {\em complete}
is the set of mapped functions is a spanning set, otherwise the
ALU is {\em incomplete} or {\em abstract}.

We wish to extend these definitions to the whole category $\Nat_{n+1}$
(the $+1$ being required in the notation so that $N_n$ is the largest type.)

We can do this as follows: first, a function with codomain $N_c$ for $c<n$
can be extended to a function with codomain $N_n$ by post composition with
the embeddding from $N_c$ to $N_n$. The beauty of this arrangement is that,
since $N_c$ is simply a subset of $N_n$, and thus an integer of $N_c$ is the
very same integer of $N_n$, the operation is merely a type cast to ensure
type correctness and has no run time impact; in other words it is has
no operation and does not require an instruction to effect it.

Secondly, if the domain of the function $f$ is $N_d$, we can replace
it with a function $g$ provided $g\restriction_{N_d}=f$ and pre
compose another embedding, which as above is a type cast with no
run time impact.

Indeed, this second method subsumes the first if $g:N_n\rightarrow N_n$.

Thus we can represent functions of the whole category, 
provided we have suitable function on the largest type.
This is not a universal property of categories of course,
but peculiar to a specific construction of the $\Nat$ family.

It is well known, for any category, that given any chain of arrows,
appropriate identity arrows can be added at any point, without changing
the resulting composite, and, identities can be removed at any point,
unless the chain consists of a single identity.

However for the $\Nat$ family we have a much stronger result: for any
category $\Nat_n$ the chain of arrows an $\ALU$ can implement embedding
functions required to ensure type correctness can be ignored when constructing
the corresponding instruction stream for the ALU.
This is called {\em type erasure}. In effect it means that instructions
are {\em generic} in the sense that each instruction can represent a whole
family of functions.

It is important to note that more than one $g$ can represent $f$, because
the action of $g$ on values outside the domain of $f$ can be anything,
since, when $g$ is used to replace $f$, the argument to $g$ is guarranteed
to be in the domain of $f$ and the result of applying it to values
outside the domain are irrelevant, since such application cannot occur.

Another way to view this result is to say that an ALU for a category
$Nat_n$ can perform all the {\em supported} operations for all the contained
categories. The qualification {\em supported} is required in case
the ALU can only represent an abstraction; that is, if the instructions
do not form a spanning set for the largest type of the largest category.

A particular mapping of functions 
$$\mathrm{Repr}: \Hom_n(\_,\_) \rightarrow \Hom(N_n, N_n)$$
used to represent them is called a {\em representation} of 
the category $\Nat_{n+1}$ and in effect reduces the design
space of the ALU to the monoidal subcategory with only
the largest type retained.

\begin{example}
Consider an ALU of size $2^{64}$ with instructions
\begin{verbatim}
LDA #999    // load immediate
CLRH        // clear high 32 bits
ADD         // add high 32 bits to low 32 bits
NEG         // negate
MUL         // multiply high 32 bits with low 32 bits
\end{verbatim}
These instructions suffice to implement all the operations
of a ring of size $2^{32}$. In particular \verb$CLRH$ mathematically
implements modulo $2^{32}$. Note that the ALU can {\em NOT} perform
ring operations of size $2^{64}$ because addition and multiplication
both requires two arguments, which are represented in a single value
as a compact product.
\end{example}

\begin{example}
{\em Register files.} Most real world processors have multiple registers and
instructions which operator on them, for example one might have 8 registers
of size 32 bits. Although our ALU model only allows a single register,
we can easily consider it to be the compact product of 8 x 32 bit types
and design our instruction set accordingly.
\end{example}

\begin{example}
{\em Stack machine.} Many VM in the crypto world use a stack instead
of registers. In conventional applications this can be inefficient,
but the simplification of the instruction set signicantly reduces
the complexity of the associate AIR constraints. The run time performance
is of little interest since ZK proof generation dominates processesing.
Clearly, our ALU model allows for the use of a stack instead of, or as
well as, registers in much the same way. However the stack must be bounded
to a small size.
\end{example}

\section{The MMU}
Although in theory there is no limit to the size of an ALU,
in practice one step calculations with very large integers
are intractable; and this observation extends to computations
unpacked to any fixed number of composition steps.

At the heart of the problem is the representation of projections:
for a very large integer maintaining state, the divisor of an
arbitrary projection is roughly the same size as the state itself.
Similarly, to obtain an arbitrarily large chunk of the state,
the modulus is similarly sized.

Luckily, we already have a solution: if we are willing to throw
out some precision, we can use a logarithm for the divisor an
modulus, and have the projection operator implicitly raise them.
This is what \verb$aprj$ does. 

In most conventional machines today, a single array of base
type $2^8$, called a {\em byte} is used, with $2^{32}$ or $2^{64}$
index values, and this array is called {\em main memory}.
The logarithms, base $2^8$, used for projections, are called 
{\em addresses}, the sequence of subsequent $k$ bytes is called
an {\em object} where $k$ is the logarithm of the projection modulus.
The combination of a run time address with a compile time size is 
commonly known as a {\em pointer}.

The key difficulty here is that the digital logarithm loses information;
that is, it is not injective. However, quite apart from the large values
which can be handled, and desipite the loss of compactness resulting from
the loss of injectivity, logarithms have a major advantage, to the point
where very few current programming languages have any notion
of compact sums and products at all: division, a notoriously expensive
operation, can be replaced by subtraction, which is very much cheaper.

Most modern memory systems are little endian (just to confuse you)
so in fact we use addition: every C programmer knows that to unpack
a struct you can simply add the offset in bytes of the component.

However this leads to a complication in our maths: we now have
{\em two} kinds of products, our original compact products,
and now our non-compact ones as well. Furthermore, the programmer
no has to make choices about which kinds of products to use.

\begin{example}
An array of 64 boolean values has two viable representations:
as a single 64 bit integer, or as an array of 64 bytes. In the C++
standard arrays of objects are used for all base types except 
\verb$bool$ which has a template specialisation using a compact
representation. In fact this was an extremely poor choice because
not all the usual array operations work, which means parametric
polymorphism fails.
\end{example}

\begin{example}
In the Felix programming language, which has both compact and non-compact
products, the author originally tried to have the compiler make
the distinction and hide it from the programmer. Alas, the requirement
to support polymorphic operations soon showed this could not work.
In particular, non-compact data structures can be addressed with
ordinary pointers of machine word size, and whilst compact
pointers are indeed possible (and implemented!), they require
three machine words: an object pointer, along with the usual
quotient and modulus, and therefore have a distinct type.

It is, in fact, worth noting that to preserve generic behaviour
(since mathematically a product is a product), the author was
pushed into finally introducing a first class kinding system
so that type variables used in compact product types could
be constrained to accept only compact types.
\end{example}

\chapter{Compact Linear Types}
The theory presented in the previous chapter appears reasonably general,
but in fact it is somewhat crude {\em because} it starts with subranges
of integers, and uses them as indices for functors.

We now present a more general system of mind blowing expressive power.
Note that, desipte this, the representational computations are the same.

\begin{definition}
Given two functors for products and sums with the empty product, or unit,
denoted 1, and the empty sum, or void, denoted 0, then, a {\em compact
linear type} is defined recursively as follows:
\begin{enumerate}
\item 0 is a compect linear type
\item 1 is a compect linear type
\item any finite product of compact linear types is compact linear
\item any finite sum of compact linear types is compact linear
\end{enumerate}
Note that in fact the first two conditions can be dropped, since
they are subsumed by the second two conditions. 
\end{definition}

A {\em unitsum} is a special case which is a finite sum of units,
that is, a type of the form $1+1+...+1$. This is just our
original type $N_n$ for the sum of $n$ units.

However there is a significant extension to our original formulation:
where previously we used sum and product functors indexed by
$N_k$ for some $k$, we now allow {\em any compact linear type}
as the functor index type.

To understand how radical an upgrade this is, and what it actually
means, consider the product $2 \times (3 \times 4)$. Products are
not associative so we first construct $3\times4$ with a bifunctor
indexed by $N_2$, then construct the whole term, with a second
bifunctor, also indexed by $N_2$.

This is clearly inconvenient, we would like to compose the functors
so there is only one functor, and we do that {\em by changing
the index type} to $1+(1+1) = 1 + 2$.

It's important to see that the index is an exponential, and so obeys the
usual index laws. In particular the domain category is
$$\Nat^{1+2} = \Nat \times {\Nat^2}$$
which means the LHS is precisely the functor composite of the RHS.
What we have done is {\em lifted the structure out of the codomain
into the domain}. In particular, we have {\em linearised} the
structured type in doing so.

To see what this means, lets consider that the value of the RHS type
$$(1,(2,3))$$
is in fact
$$(1,2,3)$$
in the LHS type. What's more, the {\em representation} type of both
is precsiely $N_{24}$ and iterating through the representation will
find all the values, in order, for both types.

In particular, every structured type is an array, with an index
that models the type as an exponential. Consider another example:
$$((1,2,3),(4,5,6))$$
This is an array length 2, of arrays length 3; that is, it has type
$$(N^3)^2$$
However it is equivalent to an single linear array
of type
$$N^{2\times 3}$$
with the corresponding value being
$$(1,2,3,4,5,6)$$
In particular you can see when we raise an exponential to an index
we get a product, and when we raise a product to an index we get a sum.
Conversely, if the index of a functor is a sum, the type designated
is a product, and if a product it is an exponential.

To put this another we, the index is the {\em logarithm} of the type.
In our array of arrays case, we have converted the type to a matrix
so instead of writing $(x.1).2$ for a component we can now
write $x.(1,2)$ and the subcript is now a pair, instead of two
successively applied subscripts. An array of arrays is not a matrix,
because a matrix does take a single index which is a pair.

Linearisation is an isomorphism, not an identity; but it has the special
property that the underlying representation is not changed; that is, the order
of array elements is invariant. In code, this means it is a suitable static
cast, involving no run time operations. Contrast this with symmetry, which is 
also an isomorphism, but requires reordering of values in the store.

The interpretation is that:
\begin{enumerate}
\item When the functor has an index of type $k$, the array subscript has type $k$
\item The element accessed is the representation of $k$
\end{enumerate}

Here are more examples. Consider
$$(1,2,3),(4,5)$$
The type is 
$$N^3\times N^2$$
This can be represented as a linear array
$$(1,2,3,4,5)$$
by using a functor index of type
$$3+2$$
This means, you first chose the left summand of type 3 or the right
summand of type 2, which is picking which tuple you're interested in,
either $(1,2,3)$ of $(4,5)$, respectively.

In the first case, you now pick a value of type 3, and in the second,
a value of type 2. In other words, the array index is a variant.
To actually compute the array index, we simply use the representation
which is 0,1, or 2 in the first case, and 3 or 4 in the second.
So the representation value is an index of the original data structure,
but {\em linearatsed} or {\em flattened}.

Compact linear types were invented with the intention they be used
as array indices allowing polyadic array operations; that is,
operations on matrices of arbitrary rank and dimension, in a manner
independent of the rank and dimension. The rules are the generalistions
of the binary index laws:
$$\log (X^n\times X^m) = X ^ {n + m}$$
and
$$\log {X^n}^m = X ^ {m \times n}$$
Note the reversed order of the second case law. This is necessary
so that in sequencing, the $n$, or inner index moves fastest.

\cleardoublepage
\addcontentsline{toc}{chapter}{Bibliography}
\printbibliography

\end{document}

