
\documentclass[oneside]{book}
\usepackage{xcolor}
\definecolor{bg}{rgb}{0.95,0.95,0.95}
\definecolor{exbg}{rgb}{0.80,0.95,0.95}
\definecolor{emphcolor}{rgb}{0.5,0.0,0.0}
\newcommand{\empha}{\bf\color{emphcolor}}
\usepackage[framemethod=TikZ]{mdframed}

% theorem like environments
\usepackage{amsthm}
\theoremstyle{plain}
\newtheorem{example}{Example}
\theoremstyle{definition}
\newtheorem{definition}{Definition}
\theoremstyle{plain}
\newtheorem{lemma}{Lemma}
\newtheorem{theorem}{Theorem}
\newtheorem{notation}{Notation}

\mdfdefinestyle{MyFrame}{innerleftmargin=20pt}
\newenvironment{myexample}%
  {\begin{mdframed}[style=MyFrame,backgroundcolor=exbg]}%
  {\end{mdframed}}
\usepackage{parskip}
\usepackage{minted}
\usepackage{caption}
\usepackage{amsmath}
\usepackage{amssymb}
\usepackage{amscd}

% minted
\usemintedstyle{friendly}
\setminted{bgcolor=bg,xleftmargin=20pt}
\usepackage{hyperref}
\hypersetup{pdftex,colorlinks=true,allcolors=blue}
\usepackage{hypcap}

\title{Maclean\\ A new language for the modern world}
\author{John Skaller}
\begin{document}
\maketitle
\tableofcontents
\chapter{Rationale}
For many years binary byte addressable von Neumann machines dominated the computing world.
However the world has changed, with increasing demands for security and privacy and
trusted access to computing power required in an untrustworthy environment.

With the rise of cryptography and the invention of the block chain, new capabilities
have arisen, along with new requirements. The established von Neumann architectures
are no longer suitable.

Block chain operations are basically microservices operating in an untrusted environment,
where proof that a server faithfully executed a program is required, in addition to the
older requirement that the program be correct. Program correctness has been addressed
by advanced type systems, but the new breed of system, often represented by a circuit
or a virtual machine, no longer uses the same basic types as the traditional signed
and unsigned integers of 8, 16, 32 or 64 bits.

Instead, finite fields are required for cryptographic reasons, addressing models vary,
and performance is now dominated by the cost of proof generation, rather than the
cost of execution.

We still need strong static typing for proofs of correctness, but computation and
optimisation must now deal with both execution, and proof generation. Performance
of virtual machines is considerably degraded so that the proof generation becomes
tractable.

A new language is required, in which at least the basic data types are no longer
simple signed or unsigned integers and machine word size addresses, but which
can support a more general class of finite rings for conventional computations,
and finite fields of hashing and other cryptographic calculations.

In addition, the ad hoc nature of older languages is no longer tolerable,
instead we need a system soldidly based in algebra: ring theory and finite field
theory providing the basic types, and category theory providing ways to
construct new types from the base, abstraction, polymorphism, and modularity.


\section{Ring Theory}
The heart of a computer is just a ring called the ALU. We propose
a system as follows: a linear ring is a subrange of the integers
0..n-1 with the usual ring operations plus (integer) division and
modulus.

The user starts by declaring one or more rings to be {\em intrinsic}
and can then define new rings using any already defined rings, including
the intrinusics. These are just the usual operations modulo the ring size,
to handle multiplication, the base ring needs to be at least size $(n-1)^2$.

For example you can define \verb$u32$ in the instrinsic prime field
provided it is big enough, the compiler will generate all the necessary
operatiuons.

It is possible to define ring operation given only a smaller ring,
using the equivalent of high school long addition, multiplication
and division, {\em provided} we also supply a memory model for
addressing.

\section{Category Theory}
We use category theory to create new types from old ones.
Our type system must be at least a distributive category.
This gives us products (tuples and arrays) and sums (variants).

\section{Addressing}
Addressing is a difficult topic. If you consider memory to be one
huge integers, an object can be extracted by dividing by a
another large integer and then applying a modulus, these are
simply generalisation of the usual shift and mask operations without
the power of two requirement.

However the divisor must be about the same size as all of memory,
which is not tractable. How is this problem solved on a traditional
machine?

The answer is to take its logarithm, which is called an address.
However the result causes a loss of information, which then requires
memory to be broken into chunks. On a traditional machine the address
is 64 bits long and the chunk is 8 bits long. Consequently, we cannot
address fine details (such as individual bits), and we also cannot
easily span large values.

This problem is solved in C by padding. So we must do the equivalent,
only now we must forget binary coding, since our rings and fields
and addressing no longer need to be powers of 2.

Note that from the point of view of a circuit the application of an exponential
to recover the value which we previously took the logarithm of is done
using a decoder which creates a fanout dependent on the base of the logarithm,

\chapter{Type System}
We start needing four classes of types: sequences, groups, rings, and fields.
Each of these type will be considered a finite, linear, subset of the integers 
from 0 to $n-1$ where n is the size of the set.

\subsection{Preliminary definitions}
\begin{definition}
A {\em semi-group} is a set together with an associative binary operation; that is
$$(a \cdot b)\cdot c = a \cdot (b \cdot c)$$
where infix $\cdot$ is taken as the symbol for the binary operation.
\end{definition}

Associativity is a crucial property for an operator because it allows concurrent
evaluation of arbitrary subsequences of a sequence of operands. For example given
the operand expression 
$$(((1\cdot2)\cdot3)\cdot4)\cdot5$$
we can compute $1\cdot2$ and $4\cdot5$ concurrently:
$$(1\cdot2)\cdot3\cdot(4\cdot5)$$
then then combine 3 to either the LHS or RHS subterm before performing the final combination.
Another way of looking at this property is that the values to be combined can be stored in
the leaves of a tree and any bottom up visitation algorithm can be used to find the total combination.

Associativity means you can add or remove (balanced pairs of) parentheses freely.
In particular it is common practice to leave out the parentheses entirely.

\begin{definition}
A {\em monoid} is a semigroup with a unit $u$, that is
$$x\cdot u = x {\rm \ and\ }u \cdot x = x$$
for all x in the set.
\end{definition}

The existence of a unit means you can freely add or remove units from
anywhere in your computation.

\begin{definition}
A {\em group} is a monoid in which every element has an inverse, that is,
for all $x$ there exists an element $y$ such that
$$x \cdot y = u {\rm\ and\ } y \cdot x = u $$
\end{definition}
where $u$ is the unit of the underlying monoid. For integers of course,
the additive inverse of a value is its negation.

\begin{definition}
An operation is {\em commutative} if the result is the same with the operands
reversed, that is, for all $a$ and $b$.
$$a \cdot b = b \cdot a$$
A group is said to be commutative if the group operation is commutative.
\end{definition}

Commutativity says you can switch the order of children in the tree representation
of an expression.

If an operation is also associative, commutative, and has a unit, then the operation
is well defined on a set of operands, taking the operation on the empty
set to be the unit. 

This means irrespective of what data structure you use to hold the
values to be combined, and what algorithm you use to scan them,
provided you visit each value exactly once, the result of the
operation on them is invariant.

\begin{definition}
A {\em ring} is a set with two operations denoted by $+$ and $*$ such
that the set with $+$ is a group, and the set excluding the additive
unit is a monoid, and the following rule, called the
{\em distributive law} holds for all $a$, $b$ and $c$
$$a * (b + c)  = a * b + a * c$$

If the multiplication operation is commutative then it is called
a commutative ring.
\end{definition}

\subsection{The rings $\mathbb{N}_n$}
\begin{definition}
Let $\mathbb{N}_n$ be the subrange of the integers $0..n-1$ with 
addtion, subtraction, 
multiplication, division and remainder defined as the natural result modulo $n$.
Then $\mathbb{N}_n$ is a commutative ring called a {\em natural ring}.
\end{definition}

The usual linear order is also defined.  Negation is defined by
$$-x = n - x$$

Natural computations prior to finding the modular residual present an issue
we resolve by performing these computations in a much larger ring.

\begin{definition}
The {\em size} of a finite ring $R$, written $|R|$, is the number of values of the underlying set.
\end{definition}
\subsection{Representation}
\begin{lemma} The C data types
\begin{minted}{C++}
uint8_t uint16_t uint32_t uint64_t
\end{minted}
with C builtin operations for addition, subtraction, negation, and multiplication
are the rings
\(\mathbb{N}_{2^8}\ \mathbb{N}_{2^{16}}\ \mathbb{N}_{2^{32}}\ \mathbb{N}_{2^{64}} \)
respectively, with the usual comparison operations, unsigned integer division,
and unsigned integer modulus.
\end{lemma}

\begin{theorem}
{\em Representation Theorem}. The values of a ring $\mathbb{N}_n$ can be represented
by values of a ring $\mathbb{N}_{n^2}$ and the operations addition, substraction, negation
multiplication and modulus computed by the respective operations modulo $n$. Comparisions
work without modification.
\end{theorem}
In particular we can use \verb$uint64_t$ to represent rings of index up to 
$2^{32}$.

\section{Intrinsics}
An instrinsic is a ring or field which is provided natively by the target system.
We use the following example to show how:
\begin{minted}{felix}
field goldilocks = 
  intrinsic size = 2^64 - 2^32 - 2, 
    add = primitive cost 1,
    neg = primitive cost 1,
    sub = fun (x,y) => add (x, neg y),
    mul = primitive cost 1,
    udiv = primitive cost 4,
    umod = primitive cost 4,
    udivmod = primitive cost 4,
    recip = primitive cost 1,
    fdiv = fun (x,y) => mul (x, recip x),
    ... // TODO: finish list
;
\end{minted}
Each of the required operations for a field (or ring if a ring is being specified),
must be either implemented in the target natively, in which case the number of
execution cycles required must be specified, or is defined in terms of another
defined operation, in such a way no cyclic dependencies exist. In the latter case
the compiler derives the cost from the definition.

\subsection{Versions}
In version 1, a definition must be given first before it can be used.
In later versions, the a dependency checker will ensure completeness
and consistency. In still later versions defaults may be provided.

\section{Derived structures}
\subsection{One step derivations}
Once we have one or more intrinsic, we can use the type notation
\begin{minted}{felix}
N<n, base>
\end{minted}
to specify a ring of size n, defined using the already defined ring \verb$base$.
The compiler will define all operations automatically, using modular arithmetic
etc, provided $n^2\leq |{\mathtt base}|$, otherwise it will issue a diagnostic
error message that the base ring is not large enough and terminate the compilation.

Note, the base ring does not have to be intrinsic.

To define a new field, we need only a ring sufficiently large for the underlying
ring operations, however we must define the \verb$recip$ operation natively:
\begin{minted}{felix}
field nufield = based goldilocks recip = primitive cost 24;
\end{minted}

\subsection{Families}
All data types derived directly or indirectly on a particular
base for a family, Operations with mixed families require
the specification of isomorphism between abstractly equivalent types,
or embeddings if approproiate. Research is required here to decide
how to handle computations with mixed families.
\begin{minted}{felix}
to be done
\end{minted}

\section{Compact Products}
We first provide {\em compact linear products}. This is a categorical product
with the type given by the n-ary constructor like
\begin{minted}{felix}
compactproducttype<R0, R1, ... Rnm1)
\end{minted}
and values like
\begin{minted}{felix}
compactproductvalue(v0,v1,... vnm1)
\end{minted}
Note syntactic sugar is yet to be determined.

All the data types must be in the same family, and the product
of their sizes must be less than or equal to the family base size.
All the operators are defined componentwise. However sequencing
is based on the representation.

Projections and slices are provided automatically
\begin{minted}{felix}
compactprojection<index, base> : T -> Rindex
\end{minted}
This is a function which extracts the component selected
by the index. The index must be a constant.

A generalised projection is also provided which accepts an
expression as an argument. Its codomain is the type dual
to the product, that is a sum type consisting of the component
index and value.  We note that the representation is uniform because
the constructor arguments are all structures from the same family.

\section{Arrays}
If all the types of a product are the same, it is called an array.
An array projecton accepts an expression as an argument, which cannot
be out of bounds, since the type of the expression must be the index
type of the constructing functor. The array projection is defined by
first applying a generalised projection and then applying the smash 
operator which throws out the constructor of a repeated sum.

Note there is a related operator which retains the index value,
so that an iteration will get a index,value pair.

The math is straightforward but the syntax needs to be established.

\section{Sums}
Just as for products, we provide categorical sums including
repeated sums, which are the dual of arrays, along with 
injections functions.

However the sum of rings is not a ring in the category or rings.
In the category of types, the operations on the sum of two rings
is instead define by the operations on the representation.

Since our rings are cyclic, the sum of N<3> and N<4> behaves like
N<7>. However note, it is still a proper categorical sum, since we
can decode it to extract a value of one of the injection types.

In a sum of rings, addition is precisely addition with carry,
and multiplication is modulo the size of the sum (which is the
sum of the component sizes).

\chapter{Modularity}
Contrary to popular belief and design, an appropriate unit of modularity
is the {\em routine.}

A basic routine is a labelled lexically contiguous sequence of instructions
accepting an argument and with a terminal operation which either halts
the routine, or invokes another.

The terminal is a sum type in which the constructors names must be labels
of routines, and in which the arguments of the constructor become the
argument of the invoked routine.  The exit selected is determined by
the injection which created the value of the sum type.

In a more conventional and relaxed form, a routine can contain multiple
exits possibly selected by a conditional.

In this model, a {\em subroutine} is a special case of a routine in which
the caller passes it's current continuation (the next instruction) as, or as
part of, the called routine's argument implicitly. Then the caller can be resumed
by use of a return instruction.

A {\em function} is a special case of a subroutine, in which the caller's suspension
contains an unitialised reserved slot for a value which is assigned a value
by the return instruction before the called routine terminates and the caller
is resumed.

A coroutine is a different special case of a routine which is passed a continuation,
but not necessarily the callers continuation. The coroutine can then exit by
resuming any suspension.

The usual way to organise coroutines is using a schedular and channels in the style
of Tony Hoares communication sequential processes (CSP) but with indeterminate
evaluation replacing concurrency.

Central to the execution model is the idea that a routine can suspend itself,
and be resumed by others, in particular the notion of continuation passing
is fundamental, and functions are merely a special case.

In particular, functions are very weak at handling failure, for example
division by zero, so a routine that performs a division should in fact
have two continuations to choose from.  in this case favouring one
as special, which is what happens with a subroutine due to the implicit
acceptance of the callers continuation makes the divide by zero exceptional,
which is clearly bad design.

\chapter{Addressing}
One of the core tenants of a von Neumann architecture is that memory is
addressable, and programs are stored in memory.

Standard ISA usually involve bytes and words which are rings of size powers of $2^8$,
and there will be a word size, typically a ring $2^64$ which is used to address memory.

We must generalise this model, because many modern VM use finite fields as basic
memory elements, and such fields never have a size which is a power of 2 since
for a ring to be a field, it must be a product of rings of a size which is a prime number,
and 2 is too small for most calculations.

This all leads to the question, exactly what is an address?


\end{document}

