
\documentclass[oneside]{book}
\usepackage{xcolor}
\definecolor{bg}{rgb}{0.95,0.95,0.95}
\definecolor{emphcolor}{rgb}{0.5,0.0,0.0}
\newcommand{\empha}{\bf\color{emphcolor}}
\usepackage{parskip}
\usepackage{minted}
\usepackage{caption}
\usepackage{amsmath}
\usepackage{amssymb}
\usepackage{amscd}
\usemintedstyle{friendly}
\setminted{bgcolor=bg,xleftmargin=15pt}
\usepackage{hyperref}
\hypersetup{pdftex,colorlinks=true,allcolors=blue}
\usepackage{hypcap}

\title{Synchronous Runtime}
\author{John Skaller}
\begin{document}
\maketitle
\tableofcontents
\chapter{Overview}
The Felix system runtime library contains a subsystem for synchronous scheduling of fibres.
This system is written in C++ and can be used by suitable C++ programs.

This document describes how a C++ programmer can utilise the synchronous fibre scheduler.

\section{Components}
A synchronous systems consists of the following components:

\begin{itemize}
\item A scheduler object
\item A collection of fibre objects
\item A collection of channel objects
\end{itemize}

Whilst the synchronous scheduler is operating, there is exactly one
fibre which is curretly running, and a set of active fibres which are ready to
run, possibly empty. Active fibres are in a suspended state.

A fibre may take action which causes it to be suspended or terminate 
so that it is no longer running. In this case it returns control to the scheduler which
picks another active fibre to run. If there are no active fibres, the scheduler
has no work to schedule and returns to its caller.

A fibre may take action which cases another fibre to become active.
There are two ways it can do this:

\begin{itemize}
\item Spawn another fibre
\item Complete a synchronous I/O operation
\end{itemize}

When a fibre is spawned it is added in a suspended state to the set of
active fibres as is the spawner. The scheduler then picks an active
fibre to run. 

In the current implementation, the spawned fibre is always picked,
so that spawning a fibre acts in the same way as calling a subroutine.

I/O operations are performed on channels. There are two I/O operations.
\begin{itemize}
\item Read a machine word from a channel
\item Write a machine word to a channel
\end{itemize}

A channel is a set of fibres, possibly empty, all of which have performed
a read or all of which have performed a write. 

When a read is performed
by a fibre on a channel, if the channel consists of readers, the requesting
fibre is suspended and added to the channel. If the channel contains a writer,
a machine word is transfered from the writer to the reader, the writer
is removed from the channel, and the reader and writer are made active.
The scheduler then picks an active fibre to run.

When a write is performed
by a fibre on a channel, if the channel consists of writers, the requesting
fibre is suspended and added to the channel. If the channel contains a reader,
a machine word is transfered from the writer to the reader, the writer
is removed from the channel, and the reader and writer are made active.
The scheduler then picks an active fibre to run.

In the current implementation the reader is always picked by the scheduler
to run. In the current implementation the channel is a singly linked list
of fibres so that if an I/O request can be immediately satisfied, it is
satisfied by the most recent matching operation, that is, in FILO or
stack order.

When an I/O operation on an schannel adds the requesting fibre to the
channel it is said to be unsatisfied. When a read removes a writer,
or a write removes a reader, the operation is satisfied and the reader
and writer are said to be matched.





\end{document}

\documentclass[oneside]{book}
\usepackage{xcolor}
\definecolor{bg}{rgb}{0.95,0.95,0.95}
\definecolor{emphcolor}{rgb}{0.5,0.0,0.0}
\newcommand{\empha}{\bf\color{emphcolor}}
\usepackage{parskip}
\usepackage{minted}
\usepackage{caption}
\usepackage{amsmath}
\usepackage{amssymb}
\usepackage{amscd}
\usemintedstyle{friendly}
\setminted{bgcolor=bg,xleftmargin=15pt}
\usepackage{hyperref}
\hypersetup{pdftex,colorlinks=true,allcolors=blue}
\usepackage{hypcap}
\title{Type Systems}
\author{John Skaller}
\begin{document}
\maketitle
\tableofcontents
\part{Subtyping Kernel}
\chapter{Motivation}

