
\documentclass{article}
\usepackage{xcolor}
\definecolor{bg}{rgb}{0.95,0.95,0.95}
\definecolor{emphcolor}{rgb}{0.5,0.0,0.0}
\newcommand{\empha}{\bf\color{emphcolor}}
\usepackage{parskip}
\usepackage{minted}
\usepackage{caption}
\usepackage{amsmath}
\usepackage{amssymb}
\usepackage{amscd}
\usepackage{makeidx}
\makeindex
\usemintedstyle{friendly}
\setminted{bgcolor=bg,xleftmargin=20pt}
\usepackage{hyperref}
\hypersetup{pdftex,colorlinks=true,allcolors=blue}
\usepackage{hypcap}
\newcommand*{\fullref}[1]{\hyperref[{#1}]{\autoref*{#1} \nameref*{#1}}}
\DeclareMathOperator{\quot}{div}
\DeclareMathOperator{\rmd}{rmd}
\newtheorem{theorem}{Theorem}
\newtheorem{lemma}{Lemma}
\newtheorem{definition}{Definition}
\newtheorem{notation}{Notation}
\def\restrict#1{\raise-.5ex\hbox{\ensuremath|}_{#1}}


\title{LLVMN to Midden\\ Proof of Princple Project}
\author{John Skaller\\ mailto:skaller@internode.on.net}
\date{8 Dec 2022}
\begin{document}
\maketitle
\section{Purpose}
The purpose of this document is to outline a proposal to develop technology which
demonstrates translation of LLVM assembler to Midden Assembler is possible,
and discover restrictions on the inputs to allow the translation to proceed,
and enhancements required in order to cover a significant part of the Midden
VM capabilities.

\section{Deliverables}
\subsection{The executable}
Source code and build instructions to build an executable which can translate a
single LLVM assembler source file into an Midden assembler source file.

\subsection{Usage Description Document}
Should specify:
\begin{enumerate}
\item How to use the executable to perform the translation.
\item How to assmeble the translation and execute the result on a Midden VM.
\item How to obtain the result and verify it is as expected.
\item How to obtain LLVM assembler from inputs in C and Rust.
\end{enumerate}

\section{Input Requirements}
\begin{enumerate}
\item Input test cases written by hand in LLVM assembler.
\item Input test cases written in C.
\item Input test cases written in Rust.
\end{enumerate}
The input tests need two forms. The translator developer will write some unit level
test cases in LLVM assembler and a few more complex functional cases; these
will not be executable on Midden but instead just demonstrate the translation has proceeded
as expected. Verification will by visual examination.

Experts in Midden and the kinds of small operations it is expected
to perform will need to write and supply additional test cases
intended to actually execute on Midden, along with the expected behaviour.
If these are written in C or Rust, the commands to invoke the front end
translator of LLVM and produce assembler output must be provided.

\section{Operating environment}
For test purposes, both MacOS and Linux must be supported.

\section{Translator Structure}
The translator will need to operate in three phases.
\begin{enumerate}
\item The first phase, the {\em translator front end} will parse 
LLVM assembler into an internal encoding
used as inputs by the second stage.
\item The second phase translates this internal input format
into an internal output format by various processes including
traditional term reductions (similar in concept to beta reductions
in the lambda calculus).
\item The final phase, the {\em translator back end} will print out
the Midden Assembler as a text file.
\end{enumerate}

\section{Translator programming language}
The translator needs to be written in some programming language.
Options are roughly:
\begin{enumerate}
\item C++
\item Felix
\item Ocaml
\item Rust
\item Haskell
\end{enumerate}

Since LLVM is written in C++, it has the advantage the future integration
with LLVM, that is, making MiddenVM a standard LLVM back end, may
be easier. However C++ is a very poor choice for writing a compiler.

Rust is a better choice however this author is not an experience Rust
programmer. Down the track this may be more viable since many developers
involved in the Midden project use Rust. However Rust memory management
is specifically designed to avoid the need for garbage collection which
is very big advantage in developing translators.

Haskell would be an excellent choice especially for the middle phase
of the translation, it has all the facilities required. However the
author is not an experienced Haskell programmer and the purely 
functional model can sometimes be an impediment.

Ocaml is a good choice, since the author is familiar with it,
and the Felix compiler is written in Ocaml, so the author is familiar
with the techniques for writing compilers in Ocaml. However not many
developers know Ocaml.

Felix is an interesting possibility. It is capable of both procedural
and functional programming, has a strong static type system, although
not as strong as Ocaml or Haskell, but much better than C++, and it
is garbage collected. Unfortunately only the author, being the prime
developer of Felix, is familiar with it.

Felix has another advantage: it generates C++ and is specifically designed
to embed C++ too. So future integration with LLVM may be easier and large
numbers of useful libraries are available. For example Google RE2 is actually
already part of the run time.

It can also be run like Python, with no need for any build systems.
It has much better string handling than Ocaml, and is not so heavily
oriented to purely functional programming.

\section{Intellectual Design Requirements}
The principal intellectual problem to be solved and demonstrated 
by the initial work is to find an architectural mapping between the two machine models.

\begin{enumerate}
\item Memory access model
\item Instruction mapping
\item Function call mapping
\end{enumerate}
 
Although both LLVM and Midden are stack machines, 
LLVM code assumes instructions
can access arbitrary local variables in a function stack frame,
and uses three address code to represent operations.

Midden on the other hand uses zero address code to perform 
operations and cannot access most of the stack using offsets
(except the first few words).

The translator therefore needs a model to keep track of where
local variables are on the stack and a way to retrieve them,
a process which itself modifies the location of the variables with
respect to the current stack pointer. So whilst the mapping
of three address instructions such as integer addition from LLVM
to pair of loads followed by a zero address add is actually 
fairly straight forward, modelling LLVM's random access linear memory
model correctly is a significant problem and doing so efficiently
is even harder.

In addtion, Midden has some specialised functions for hashing and
other operations which have no direct equivalent in LLVM. So a way
to invoke these instructions must be specified, probably by
using an LLVM function call. The translator will then see these
particular calls and invoke wrapper code, hand written in Midden
assmbler, which actually performs the operation on the required
arguments.

Finally, LLVM does not specify directly how a function is called.
Instead it provides a simple universal call syntax with annotations
suggesting which callling protocol should be used. A calling protocol
in Midden is likely to be
\begin{enumerate}
\item Push each argument onto the stack
\item call the function
\item the return value is left on the stack.
\end{enumerate}

where the significant challenge is getting the arguments
specified by a LLVM stack offset onto the top of the Midden
stack where instructions can process them.

Another point that needs investigation is whether use of operations
on the Midden base prime field are simply using the field as a local
approximation to the integers, or actually require respecting the
field operations (especially division). Most {\em ordinary} programming
wants either natural numbers or integers and small operations on these
can be performed in a suitably large ring, so the exact inverses of
values are not required. Those inverses are, however, required
in many cryptographic operations.

Languages like C do not provide prime field operations natively
although they can be encoded with basic arithmetic. The problem here
is deciding when a formula using manual prime field encoding,
such as exact field division, is can be replaced by Midden VM
prime field operations, and when ordinary arithmetic is intended.

It's likely suppot with be needed in high level language sources
to allow the distinction to be made by the programmer in these 
sources, and allow the choices to flow through into the LLVM
assembler in such a way the translator can choose the correct
translation model.

\end{document}
