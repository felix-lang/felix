\documentclass{report}
\title{Felix Compilation Model}
\author{John Skaller}
\begin{document}
\maketitle
%\tableofcontents
\chapter{Command Line Interface}
The Felix compiler is designed to be used in conjunction with a host management
program, various Felix system libraries, and user supplied application support
libraries. The system is fully integrated by the command line processor 
\verb%flx% which organises the compilation, linking, and execution of a 
Felix script file. The system is designed to have the look and feel of a
conventional scripting language such as Perl or Python, driving the Felix compiler,
C++ compiler and linker without need for complex options or specification of
resource locations.

\section{Basic operation}
Most programs can be run simply by:
%
\begin{verbatim}
flx filename
\end{verbatim}
%
or equivalently
%
\begin{verbatim}
flx filename.flx
\end{verbatim}

On Unix, if the first line of the file is
%
\begin{verbatim}
#!/usr/bin/env flx
\end{verbatim}
%
and the file is marked as executable, then the script can be run just by
invoking the filename:
%
\begin{verbatim}
filename.flx
\end{verbatim}

A similar facility is available for Microsoft Windows.

\section{Selecting Felix Version}
Felix support an arbitrary number of independent Felix installations.
All of the program files, apart from the command line processor \verb%flx%, 
can be installed in a single directory. This directory will typically
be named for the Felix version, for example:
%
\begin{verbatim}
felix-1.1.4_rc4
\end{verbatim}
%
describes Felix version 1.1.4, release candidate 4. On Unix these are
typically installed in one of
%
\begin{verbatim}
/usr/lib
/usr/local/lib
/opt/felix
${HOME}/lib
\end{verbatim}
%
or some such place where packages are installed. The \verb%flx% script
will usually have a hard coded constant refering to the installation location.

\subsection{Overriding the version on the command line}
This can be overridden by specifying the installation directory using the
\verb%--test% option:
%
\begin{verbatim}
flx --test filename
flx --test=/opt/felix-1.1.4_rc4 filename
\end{verbatim}
%
The first variant specifies the current directory, and is primarily used by
developers testing the system. The second variant can be used to specify a 
particular installation, allowing an upgraded system to be installed without
compromising existing working programs.

\subsection{Overriding the version with environment variables}
Felix also provides an environment variable \verb%FLX_INSTALL_DIR% 
which can be used to override the installation directory:
%
\begin{verbatim}
FLX_INSTALL_DIR=/opt/felix-1.1.4_rc4 flx filename
\end{verbatim}

On Unix, this can also be set in a program file like:
%
\begin{verbatim}
#!/usr/bin/env FLX_INSTALL_DIR=/opt/felix-1.1.4_rc4 flx
\end{verbatim}
%
to ensure a program is run with a particular version.

\subsection{Querying the Version}
The version of the \verb%flx% script can be found by:
%
\begin{verbatim}
flx --version
\end{verbatim}
Note that this is the version of the \em{flx script} and not
the version of the installation!

\subsubsection{Alternate driver scripts}
Felix provides three implementations of the driver script:
\verb%flx% is written for Unix \verb%bash%, \verb%flx.bat% is written for Windows
\verb%CMD.EXE% shell, and \verb%flx.py% is written for Python and should operate
on all platforms where Python is installed.

\subsubsection{Getting Help}
In GNU tradition, some help on the command line can be obtained by
simply typing
%
\begin{verbatim}
flx
\end{verbatim}
%
which will reply
\begin{verbatim}
usage: flx [options] filename
\end{verbatim}
%
More extensive help can be obtained on Unix by 
\begin{verbatim}
flx --help
\end{verbatim}
%
which invokes the \verb%man% processor for the main
Felix man page. 

\subsection{Linkage model}
Felix supports two linkage models: dynamic and static.
Dynamic linkage is the default if the target platform supports it.
Some platforms, such as embedded systems, do not support dynamic
linkage. Other platforms, notable Microsoft Windows, and increasingly
Linux systems, do not provide static link versions of important libraries,
and so dymamic linkage is mandatory.

\subsubsection{Static Linkage}
In the static linkage model, Felix generates C++ code which is compiled
to an object file, extension \verb%.o% on Unix or \verb%.obj% on Windows.
This file is then linked against various libraries, typically \verb%.a%
files on Unix, or \verb%.lib% files on Windows. All Felix built static
linkage files have the suffix \verb%_static% added to the library name,
for example:
%
\begin{verbatim}
lib/rtl/libflx_gc_static.a
\end{verbatim}
provides the garbage collector library in static link form on Unix.
Note the prefix \verb%lib% which Unix favours. On Cygwin the prefix
is \ver%cyg%, whereas Windows does not use a prefix.
%
The \verb%_static% suffix is essential to prevent \em{smart} linkers
such as Unix \verb%ld% from linking with either static or dynamic linkage
depending on availability, and thereby producing unexpected program form,
corrupted programs, or subsequently failing for hard to understand reasons
when other libraries are not available in static link format.

Static linkage is invoked by
\begin{verbatim}
flx --static filename
\end{verbatim}
%
and creates a statically linked executable program.
Static link programs are sometimes faster than dynamic linked
programs, and are easier to ship. However they're harder to upgrade
to work with new system libraries. Note some platforms (such as Linux)
do not statically link the C library even in the static link model.
Other platforms always statically link thunks, even when linking
dynamically, notably Windows.

Finally note some Felix features such as dynamic loading of
plugins do not work with the static link model.

\subsubsection{Dynamic link model}


\end{document}
